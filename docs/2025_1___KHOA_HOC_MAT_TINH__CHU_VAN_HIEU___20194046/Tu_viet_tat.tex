%\makeglossaries
\makenoidxglossaries

% =========================
% DANH MUC THUAT NGU / TU VIET TAT
% =========================

\newglossaryentry{API}{
	type=\acronymtype,
	name={API},
	description={Giao diện lập trình ứng dụng (Application Programming Interface)},
	first={Giao diện lập trình ứng dụng (Application Programming Interface -- API)}
}

\newglossaryentry{TSJS}{
	type=\acronymtype,
	name={TS/JS},
	description={TypeScript/JavaScript},
	first={TypeScript/JavaScript (TS/JS)}
}

\newglossaryentry{UI}{
	type=\acronymtype,
	name={UI},
	description={Giao diện người dùng (User Interface)},
	first={Giao diện người dùng (User Interface -- UI)}
}

\newglossaryentry{BE}{
	type=\acronymtype,
	name={BE},
	description={Tầng xử lý phía máy chủ (Backend)},
	first={Tầng xử lý phía máy chủ (Backend -- BE)}
}

\newglossaryentry{FE}{
	type=\acronymtype,
	name={FE},
	description={Tầng giao diện phía người dùng (Frontend)},
	first={Tầng giao diện phía người dùng (Frontend -- FE)}
}

\newglossaryentry{JSON}{
	type=\acronymtype,
	name={JSON},
	description={Định dạng dữ liệu JavaScript Object Notation (JSON)},
	first={JavaScript Object Notation (JSON)}
}

\newglossaryentry{utilityfirst}{
	type=\acronymtype,
	name={utility-first},
	description={Hướng tiếp cận CSS dựa trên lớp tiện ích (Utility-first CSS)},
	first={Utility-first CSS (utility-first)}
}

\newglossaryentry{HTMLJSX}{
	type=\acronymtype,
	name={HTML/JSX},
	description={HTML (HyperText Markup Language) và JSX (JavaScript XML)},
	first={HyperText Markup Language / JavaScript XML (HTML/JSX)}
}

\newglossaryentry{eventdriven}{
	type=\acronymtype,
	name={event-driven},
	description={Mô hình hướng sự kiện (Event-driven architecture)},
	first={Event-driven architecture (event-driven)}
}

\newglossaryentry{nonblockingio}{
	type=\acronymtype,
	name={non-blocking I/O},
	description={Cơ chế I/O không chặn (Non-blocking Input/Output)},
	first={Non-blocking Input/Output (non-blocking I/O)}
}

\newglossaryentry{CSDL}{
	type=\acronymtype,
	name={CSDL},
	description={Cơ sở dữ liệu (Database)},
	first={Cơ sở dữ liệu (Database -- CSDL)}
}

\newglossaryentry{JWT}{
	type=\acronymtype,
	name={JWT},
	description={Chuẩn token xác thực JSON Web Token (JWT)},
	first={JSON Web Token (JWT)}
}

\newglossaryentry{CRUD}{
	type=\acronymtype,
	name={CRUD},
	description={Bốn thao tác dữ liệu: Create--Read--Update--Delete},
	first={Create--Read--Update--Delete (CRUD)}
}

\newglossaryentry{SEO}{
	type=\acronymtype,
	name={SEO},
	description={Tối ưu hóa công cụ tìm kiếm (Search Engine Optimization)},
	first={Search Engine Optimization (SEO)}
}

\newglossaryentry{PrismaORM}{
	type=\acronymtype,
	name={Prisma ORM},
	description={Công cụ ORM Prisma (Prisma Object-Relational Mapping)},
	first={Prisma Object-Relational Mapping (Prisma ORM)}
}

\newglossaryentry{CBF}{
	type=\acronymtype,
	name={CBF},
	description={Gợi ý dựa trên nội dung (Content-based Filtering)},
	first={Content-based Filtering (CBF)}
}

\newglossaryentry{TFIDF}{
	type=\acronymtype,
	name={TF--IDF},
	description={Phương pháp biểu diễn văn bản dựa trên trọng số Term Frequency--Inverse Document Frequency},
	first={Term Frequency--Inverse Document Frequency (TF--IDF)}
}
