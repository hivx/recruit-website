\documentclass[../DoAn.tex]{subfiles}
\begin{document}
\section{Đặt vấn đề}
\label{section:1.1}
Trong những năm gần đây, thị trường lao động đã chịu sự tác động mạnh mẽ của quá trình chuyển đổi số và sự phát triển nhanh chóng của công nghệ thông tin. Số lượng doanh nghiệp có nhu cầu tuyển dụng ngày càng gia tăng, kéo theo sự đa dạng về vị trí việc làm và yêu cầu kỹ năng. Tuy nhiên, song song với đó, người lao động, đặc biệt là sinh viên mới tốt nghiệp và người tìm việc, cũng gặp không ít khó khăn trong việc tiếp cận các cơ hội việc làm phù hợp với năng lực, kinh nghiệm và định hướng nghề nghiệp của bản thân.

Trên thực tế, hoạt động tuyển dụng hiện nay vẫn chủ yếu dựa trên các nền tảng đăng tin việc làm và phương thức tìm kiếm thủ công. Nhà tuyển dụng phải xử lý một khối lượng lớn hồ sơ ứng tuyển, trong đó không ít hồ sơ chưa đáp ứng đúng yêu cầu của vị trí tuyển dụng, gây tốn kém thời gian và nguồn lực cho quá trình sàng lọc. Ở chiều ngược lại, ứng viên thường phải tự tìm kiếm, đọc và đánh giá từng tin tuyển dụng riêng lẻ, dẫn đến việc bỏ lỡ những cơ hội phù hợp hoặc ứng tuyển vào các vị trí không thực sự tương thích.

Những hạn chế nêu trên cho thấy hệ thống gợi ý hiệu quả giữa nhà tuyển dụng và ứng viên vẫn chưa được giải quyết một cách triệt để. Nếu bài toán này được xử lý tốt, quá trình tuyển dụng có thể trở nên nhanh chóng, chính xác và minh bạch hơn, mang lại lợi ích thiết thực cho cả nhà tuyển dụng lẫn người tìm việc. Bên cạnh lĩnh vực tuyển dụng, việc giải quyết bài toán đề xuất hiệu quả còn có tiềm năng được áp dụng trong nhiều lĩnh vực khác như giới thiệu sản phẩm, hiển thị thông tin và video giải trí, nơi nhu cầu cá nhân hóa và xử lý dữ liệu ngày càng trở nên quan trọng.

\section{Mục tiêu và phạm vi đề tài}
\label{section:1.2}
Các hệ thống tuyển dụng hiện nay chủ yếu cung cấp những chức năng cơ bản như đăng tin tuyển dụng, tìm kiếm việc làm, tiếp nhận và quản lý hồ sơ ứng viên. Một số nền tảng đã bước đầu bổ sung các chức năng gợi ý việc làm hoặc đề xuất ứng viên dựa trên thông tin hồ sơ, qua đó cải thiện phần nào trải nghiệm người dùng so với phương thức tuyển dụng truyền thống. Tuy nhiên, khi so sánh tổng quan các sản phẩm hiện có, có thể nhận thấy rằng phần lớn các hệ thống vẫn tập trung vào việc hiển thị thông tin và hỗ trợ tìm kiếm ở mức cơ bản, trong khi khả năng hỗ trợ đánh giá mức độ phù hợp giữa ứng viên và vị trí tuyển dụng còn hạn chế, chưa đáp ứng tốt nhu cầu thực tế của cả hai phía.

Từ những phân tích trên, có thể khái quát rằng các hệ thống tuyển dụng hiện tại còn gặp nhiều hạn chế trong việc khai thác dữ liệu người dùng và hỗ trợ quá trình ra quyết định. Trên cơ sở đó, đề tài hướng tới việc giải quyết bài toán nâng cao hiệu quả kết nối giữa nhà tuyển dụng và ứng viên, đồng thời khắc phục những hạn chế liên quan đến đánh giá mức độ phù hợp. Phần mềm được định hướng phát triển với các chức năng chính như quản lý thông tin tuyển dụng và hồ sơ ứng viên, hỗ trợ đề xuất việc làm và ứng viên, cũng như cung cấp các cơ chế hỗ trợ tương tác và thông báo trong quá trình tuyển dụng.

\section{Định hướng giải pháp}
\label{section:1.3}
Nhằm giải quyết các vấn đề đã được xác định ở phần trước, đồ án lựa chọn định hướng xây dựng một website tuyển dụng thông minh theo mô hình ứng dụng website. Hệ thống được phát triển dựa trên hướng tiếp cận gợi ý, kết hợp việc khai thác thông tin hồ sơ và dữ liệu hành vi của người dùng trong quá trình tuyển dụng. Định hướng này được lựa chọn nhằm phù hợp với đặc thù của bài toán tuyển dụng thực tế, đồng thời tạo điều kiện cho việc mở rộng và cải tiến hệ thống trong tương lai.

Theo định hướng trên, đồ án tiến hành xây dựng một website tuyển dụng đáp ứng các chức năng chính như quản lý thông tin nhà tuyển dụng và ứng viên, đăng tin tuyển dụng, tiếp nhận hồ sơ ứng tuyển và hỗ trợ đề xuất việc làm cho ứng viên cũng như đề xuất ứng viên cho nhà tuyển dụng. Hệ thống được thiết kế với giao diện thân thiện, dễ sử dụng và phù hợp với nhiều đối tượng người dùng, góp phần hỗ trợ quá trình tuyển dụng diễn ra thuận tiện và hiệu quả hơn.

Đóng góp chính của đồ án là việc đề xuất và triển khai một mô hình website tuyển dụng thông minh hỗ trợ kết nối và sàng lọc giữa nhà tuyển dụng và ứng viên. Kết quả đạt được cho thấy hệ thống đáp ứng các yêu cầu chức năng đã đề ra, vận hành ổn định và có khả năng hỗ trợ cải thiện hiệu quả tuyển dụng trong thực tế. Đồng thời, hệ thống tạo nền tảng cho việc tiếp tục phát triển và hoàn thiện các chức năng thông minh hơn trong các giai đoạn tiếp theo.

\section{Bố cục đồ án}
\label{section:1.4}
Phần còn lại của báo cáo đồ án tốt nghiệp này được tổ chức như sau. 

Chương 2 trình bày nhu cầu thực tế và hiện trạng của các hệ thống tuyển dụng hiện nay. Trên cơ sở đó, chương này phân tích các ưu điểm và hạn chế của những hệ thống hiện có nhằm xác định hướng phát triển phù hợp cho đồ án. Kết hợp với các vấn đề đã được nêu trong Mục 1.1, chương 2 tiến hành nghiên cứu, so sánh và phân tích một số hệ thống tuyển dụng tiêu biểu để làm rõ các chức năng cần xây dựng, tạo cơ sở cho việc xây dựng các tài liệu đặc tả yêu cầu, bao gồm yêu cầu chức năng và yêu cầu phi chức năng của hệ thống.

Chương 3 được xây dựng dựa trên các yêu cầu và tính năng đã được xác định ở Chương 2, kết hợp với kiến thức chuyên môn và định hướng cá nhân của sinh viên nhằm lựa chọn và xác định các công nghệ sử dụng trong quá trình phát triển hệ thống. Chương này trình bày tổng quan về các công nghệ được lựa chọn, đồng thời phân tích vai trò, lợi ích và lý do sử dụng của từng công nghệ trong việc triển khai các thành phần của hệ thống.

Dựa trên các kết quả phân tích ở Chương 2, Chương 4 trình bày thiết kế tổng thể và thiết kế chi tiết của hệ thống, bao gồm thiết kế kiến trúc, thiết kế giao diện người dùng, thiết kế cơ sở dữ liệu và thiết kế các lớp chức năng. Chương này tập trung làm rõ cách thức hệ thống được xây dựng nhằm đảm bảo tính khả thi, khả năng mở rộng và trải nghiệm người dùng. Bên cạnh đó, Chương 4 cũng trình bày quá trình xây dựng ứng dụng, kiểm thử và triển khai hệ thống, đồng thời đánh giá mức độ đáp ứng của hệ thống so với các yêu cầu đã đặt ra ban đầu.

Chương 5 tập trung trình bày các đóng góp chính của hệ thống và các công nghệ cốt lõi được sử dụng để xây dựng một hệ thống tuyển dụng thông minh. Nội dung chương này phân tích các giải pháp và ý tưởng trọng tâm tạo nên sự khác biệt của hệ thống so với các hệ thống hiện có, đồng thời làm rõ những kiến thức và phương pháp đã được áp dụng nhằm đề xuất các giải pháp cho bài toán tuyển dụng.

Cuối cùng, Chương 6 trình bày kết quả đánh giá hệ thống thông qua việc so sánh với các hệ thống tuyển dụng tương tự. Chương này tổng hợp các nội dung mà đồ án đã thực hiện, chỉ ra những hạn chế còn tồn tại và các đóng góp đạt được, từ đó đề xuất các định hướng và hướng phát triển trong tương lai nhằm tiếp tục hoàn thiện và nâng cao hiệu quả của hệ thống.
\end{document}