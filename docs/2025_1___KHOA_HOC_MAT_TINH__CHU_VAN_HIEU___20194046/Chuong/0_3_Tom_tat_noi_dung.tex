\documentclass[../DoAn.tex]{subfiles}
\begin{document}

\begin{center}
    \Large{\textbf{TÓM TẮT NỘI DUNG ĐỒ ÁN}}\\
\end{center}
\vspace{1cm}
Trong bối cảnh thị trường lao động ngày càng cạnh tranh và chịu tác động mạnh mẽ của quá trình chuyển đổi số, hoạt động tuyển dụng theo phương thức truyền thống đang bộc lộ nhiều hạn chế, đặc biệt là trong việc tìm kiếm và lựa chọn ứng viên phù hợp với các yêu cầu đã đặt ra. Hiện nay, nhiều website tuyển dụng vẫn chủ yếu hoạt động theo mô hình đăng tin và tìm kiếm thủ công, trong đó việc kết nối giữa nhà tuyển dụng và ứng viên phần lớn dựa vào việc hai bên tự chủ động tìm kiếm lẫn nhau, chưa khai thác hiệu quả dữ liệu để hỗ trợ quá trình sàng lọc và gợi ý.

Xuất phát từ thực trạng đó, em lựa chọn thực hiện đồ án "\textbf{Phát triển website tuyển dụng thông minh}" dựa trên việc kết hợp yêu cầu của nhà tuyển dụng, mong muốn của ứng viên và hành vi thực tế của ứng viên trong quá trình tìm kiếm việc làm, từ đó đưa ra các đề xuất mang tính cá nhân hóa, phù hợp cho cả hai phía. Hướng tiếp cận này giúp ứng viên và nhà tuyển dụng có cơ sở rõ ràng hơn trong việc đưa ra quyết định, đồng thời góp phần giảm thời gian sàng lọc hồ sơ và hạn chế các vòng phỏng vấn kéo dài nhưng kém hiệu quả.

Website này đóng vai trò như một nền tảng kết nối giữa nhà tuyển dụng và ứng viên thông qua các chức năng như đề xuất công việc cho ứng viên, đề xuất ứng viên cho nhà tuyển dụng, cơ chế thông báo chủ động và các chức năng hỗ trợ tuyển dụng khác. Kết quả thu được đáp ứng đầy đủ các yêu cầu chức năng đã đặt ra, cho phép quản lý hiệu quả hồ sơ ứng viên, nhu cầu người dùng và các dữ liệu đề xuất.
\begin{flushright}
Sinh viên thực hiện\\
\begin{tabular}{@{}c@{}}
\textit{(Ký và ghi rõ họ tên)}
\end{tabular}
\end{flushright}

\end{document}