\documentclass[../DoAn.tex]{subfiles}
\begin{document}
	\section{Kết luận}
	Thị trường tuyển dụng luôn gặp vấn đề về việc nhà tuyển dụng gặp khó khăn khi tìm ứng viên phù hợp với nhu cầu và người đi ứng tuyển không tìm thấy công việc phù hợp với mong muốn, khả năng của mình. Từ đó, sinh viên cho rằng nguyên nhân quan trọng nằm ở cách tiếp cận và xử lý thông tin của cả hai phía. Đây cũng là cơ sở để xây dựng một hệ thống website tuyển dụng thông minh theo hướng đề xuất.
	
	Hiện nay trên thị trường có nhiều nền tảng tuyển dụng như TopCV, ITviec,... Các nền tảng này đem lại lợi ích lớn trong việc đăng tải và tiếp cận thông tin tuyển dụng, tuy nhiên cách tiếp cận phổ biến vẫn dựa nhiều vào mô hình đăng tin và tìm kiếm thủ công hoặc ưu tiên hiển thị theo gói dịch vụ/quảng cáo. Vì vậy, mức độ cá nhân hóa và khả năng gợi ý “đúng nhu cầu” giữa hai phía còn hạn chế, đặc biệt khi dữ liệu tăng nhanh và người dùng không có đủ thời gian sàng lọc.
	
	Hệ thống trong đồ án tập trung vào mục tiêu chính là đề xuất, do đó đây vừa là điểm mạnh vừa là thách thức. Việc ước lượng mức độ phù hợp giữa nhiều đối tượng (ứng viên -- tin tuyển dụng -- nhà tuyển dụng) với số lượng thông tin rất lớn là vấn đề khó, đòi hỏi cơ chế biểu diễn dữ liệu và công thức tính điểm phù hợp có khả năng mở rộng và giải thích được.
	
	\section{Hướng phát triển}
	Mặc dù hệ thống đã đáp ứng các yêu cầu đặt ra ban đầu và đạt được mục tiêu cốt lõi là hỗ trợ đề xuất giữa ứng viên và nhà tuyển dụng, để có thể tiến tới triển khai thực tế hoặc thương mại hóa vẫn cần một quá trình cải tiến liên tục về cả chức năng, hiệu năng và trải nghiệm người dùng. Từ kinh nghiệm trong quá trình thực hiện đồ án, sinh viên xác định hai hướng phát triển chính để tăng tính hoàn thiện và khả năng cạnh tranh của hệ thống.
	
	Thứ nhất, hệ thống hiện tại chủ yếu tập trung vào việc đưa thông tin và gợi ý phù hợp đến người dùng, nhưng chưa hỗ trợ đầy đủ một quy trình tuyển dụng trọn vẹn. Trong tương lai, hệ thống có thể mở rộng theo hướng cung cấp thêm các tiện ích và nghiệp vụ tuyển dụng, chẳng hạn như công cụ tạo và quản lý CV, hỗ trợ phỏng vấn trực tuyến, quản trị quy trình tuyển dụng theo từng doanh nghiệp, cũng như các chức năng theo dõi trạng thái ứng tuyển và đánh giá ứng viên. Việc bổ sung các thành phần này giúp hệ thống không chỉ dừng lại ở vai trò kết nối mà còn trở thành một nền tảng hỗ trợ tuyển dụng toàn diện, nâng cao mức độ gắn kết của người dùng.
	
	Thứ hai, mô hình đề xuất cần tiếp tục được nâng cấp để tăng tính thuyết phục và cải thiện trải nghiệm. Hệ thống có thể phát triển theo hướng chuẩn hóa dữ liệu đầu vào tốt hơn (đặc biệt với kỹ năng, địa điểm và mức lương), tinh chỉnh cơ chế tính điểm theo phản hồi thực tế, và tăng khả năng giải thích kết quả đề xuất để người dùng hiểu rõ lý do phù hợp. Bên cạnh đó, việc khai thác hành vi người dùng như lượt xem, lượt lưu, lượt ứng tuyển hoặc phản hồi của nhà tuyển dụng có thể giúp hệ thống dần học được sở thích và xu hướng, từ đó nâng cao chất lượng gợi ý theo thời gian.
	
	Hệ thống website tuyển dụng thông minh là kết quả của quá trình thực hiện nghiêm túc và cũng thể hiện định hướng nghiên cứu của sinh viên trong lĩnh vực đề xuất. Sinh viên kỳ vọng trong tương lai hệ thống sẽ tiếp tục được cải tiến về cả chức năng lẫn chất lượng đề xuất, qua đó đáp ứng tốt hơn nhu cầu của nhiều nhóm người dùng và phù hợp hơn với bối cảnh tuyển dụng thực tế.
	
\end{document}