\documentclass[../DoAn.tex]{subfiles}
\begin{document}
	\section{ReactTS}
	
	\subsection{React}
	\begin{figure}[H]
		\centering
		\includegraphics[width=1.0\textwidth]{Hinhve/react.png}
		\caption{Minh họa thư viện React}
		\label{fig:react}
	\end{figure}
	
	React là một thư viện JavaScript mã nguồn mở do Meta phát triển, được thiết kế để xây dựng giao diện người dùng theo hướng khai báo và dựa trên các thành phần (component). React cho phép chia nhỏ giao diện thành các khối chức năng độc lập, có thể tái sử dụng và dễ kiểm soát luồng dữ liệu, từ đó phù hợp với các ứng dụng web có nhiều màn hình và tương tác thường xuyên. [4]
	
	Trong đồ án, React được sử dụng để đáp ứng các yêu cầu ở Chương 4 liên quan đến tầng giao diện và trải nghiệm người dùng, bao gồm hiển thị danh sách tin tuyển dụng, trang chi tiết tin, trang quản lý hồ sơ ứng viên/nhà tuyển dụng, luồng ứng tuyển, và đặc biệt là màn hình hiển thị danh sách đề xuất dựa trên FitScore. Với đặc thù hệ thống có nhiều thao tác tìm kiếm, phân trang, lọc dữ liệu và cập nhật danh sách theo truy vấn, React tỏ ra phù hợp nhờ mô hình component giúp tách UI theo từng chức năng rõ ràng, đồng thời cơ chế quản lý trạng thái hỗ trợ cập nhật giao diện nhất quán theo hành vi người dùng. Cách tổ chức này cũng góp phần nâng cao khả năng mở rộng và bảo trì khi hệ thống phát triển thêm các nghiệp vụ và màn hình mới.
	
	Một số lựa chọn thay thế gồm Vue (cú pháp đơn giản, dễ tiếp cận), Angular (framework đầy đủ, phù hợp hệ thống lớn với cấu trúc chặt chẽ), hoặc Svelte (nhẹ, tối ưu hiệu năng ở runtime). Tuy nhiên, React được lựa chọn vì có hệ sinh thái phong phú, cộng đồng lớn, tài liệu và nguồn học tập đa dạng, đồng thời phổ biến trong thị trường phát triển phần mềm hiện nay, giúp thuận lợi hơn trong việc phát triển, bảo trì và mở rộng. Ngoài ra, React đồng bộ tốt với backend NodeJS do cùng hệ sinh thái JavaScript/TypeScript, từ đó giảm chi phí học công nghệ mới và giúp quá trình triển khai đồ án diễn ra hiệu quả hơn.
	
	\subsection{TypeScript}
	TypeScript là một ngôn ngữ được Microsoft phát triển dựa trên JavaScript, bổ sung hệ kiểu tĩnh (static typing) và các đặc tính hướng đối tượng nhằm tăng khả năng kiểm soát và bảo trì mã nguồn [5]. TypeScript được biên dịch về JavaScript để chạy trên trình duyệt và môi trường NodeJS, vì vậy vẫn tương thích với hệ sinh thái JavaScript hiện có nhưng cung cấp thêm cơ chế kiểm tra lỗi ngay trong quá trình phát triển. Nhờ các khái niệm như \textit{type}, \textit{interface}, \textit{generic} và khả năng suy luận kiểu, TypeScript giúp mô tả rõ cấu trúc dữ liệu và ràng buộc hợp lệ của dữ liệu trong dự án, đặc biệt phù hợp với các hệ thống có nhiều thực thể và nhiều luồng nghiệp vụ.
	
	TypeScript được sử dụng nhằm giải quyết tính ổn định và nhất quán dữ liệu giữa frontend và backend [6], đặc biệt với các cấu trúc dữ liệu có nhiều trường và dễ phát sinh lỗi như: hồ sơ ứng viên, tin tuyển dụng, vector đặc trưng (skill/tag/location/salary), và cấu trúc giải thích (explanation) trả về kèm FitScore. Khi mô hình dữ liệu thay đổi, TypeScript giúp phát hiện sai lệch kiểu dữ liệu ngay tại thời điểm biên dịch, giảm lỗi runtime và hỗ trợ refactor nhanh hơn khi hệ thống mở rộng.
	
	Các lựa chọn thay thế có thể là JavaScript thuần (phát triển nhanh nhưng khó kiểm soát lỗi kiểu khi dự án lớn dần), hoặc kiểm tra dữ liệu ở runtime bằng schema validation phía client (ví dụ: validate JSON trước khi render). TypeScript được lựa chọn vì tạo ra ràng buộc kiểu rõ ràng xuyên suốt mã nguồn, làm giảm rủi ro sai khác giữa API và UI, đồng thời giúp sinh viên quản lý tốt hơn các hợp đồng dữ liệu (data contract) trong hệ thống có nhiều luồng nghiệp vụ.
	
	\section{Tailwind CSS}
	Tailwind CSS là một framework CSS theo hướng \textit{utility-first}, cung cấp sẵn các lớp CSS nhỏ (utility classes) để xây dựng giao diện trực tiếp trong mã HTML/JSX mà không cần viết nhiều stylesheet thủ công [7]. Trong đồ án, Tailwind CSS được sử dụng để giải quyết yêu cầu ở Chương 4 liên quan đến thiết kế giao diện nhất quán, tối ưu tốc độ phát triển UI và đảm bảo khả năng mở rộng khi số lượng màn hình tăng lên. Cụ thể, Tailwind hỗ trợ xây dựng nhanh các thành phần giao diện như layout trang, lưới hiển thị danh sách, khối thẻ tin tuyển dụng, form tìm kiếm/lọc, cũng như các trạng thái hiển thị khác nhau (hover, focus, responsive) cho cả ứng viên và nhà tuyển dụng. Việc áp dụng utility classes giúp giảm trùng lặp CSS, hạn chế xung đột style giữa các module, đồng thời tạo ra một quy ước thiết kế thống nhất trong toàn bộ hệ thống.
	
	Xét về các lựa chọn thay thế, hệ thống có thể sử dụng CSS thuần kết hợp quy ước đặt tên, hoặc các thư viện UI/framework như Bootstrap, Material UI hay Ant Design. CSS thuần và SCSS cho phép tùy biến cao nhưng thường tốn thời gian xây dựng hệ thống style và dễ phát sinh thiếu nhất quán khi dự án mở rộng. Các thư viện UI như Bootstrap hoặc Material UI cung cấp sẵn component nhưng có thể làm giao diện bị phụ thuộc mạnh vào thư viện và khó đồng bộ thiết kế theo yêu cầu riêng của hệ thống. Tailwind CSS được sinh viên lựa chọn vì cân bằng tốt giữa tốc độ phát triển và mức độ tùy biến, dễ duy trì tính đồng nhất giao diện, hỗ trợ responsive đầy đủ và phù hợp với hướng phát triển theo component của React. Ngoài ra, Tailwind có thể cấu hình theme và tái sử dụng quy ước thiết kế xuyên suốt dự án, qua đó cải thiện hiệu quả triển khai và bảo trì giao diện trong phạm vi đồ án.
	
	\section{NodeJS}
	NodeJS là môi trường runtime cho JavaScript trên phía máy chủ, được xây dựng trên nền tảng V8 engine [8]. Khác với các mô hình xử lý đồng bộ truyền thống, NodeJS tổ chức xử lý theo cơ chế \textit{event-driven} và \textit{non-blocking I/O}, cho phép ứng dụng xử lý hiệu quả các tác vụ I/O như truy vấn cơ sở dữ liệu, đọc/ghi tệp, gọi API và gửi email. Nhờ đặc điểm này, NodeJS thường được sử dụng để phát triển các hệ thống web cung cấp API có tần suất truy cập cao, yêu cầu phản hồi nhanh và có nhiều thao tác bất đồng bộ.
	
	Trong đồ án, NodeJS được sử dụng để xây dựng tầng dịch vụ và API cho hệ thống, bao gồm: xác thực/ủy quyền, quản lý tài khoản, quản lý hồ sơ ứng viên và nhà tuyển dụng, quản lý tin tuyển dụng, tính toán FitScore và trả về danh sách đề xuất, cũng như kích hoạt các tác vụ gửi thông báo email. Các chức năng này có đặc trưng I/O lớn (truy vấn CSDL, đọc/ghi dữ liệu, gửi email), vì vậy NodeJS với mô hình xử lý bất đồng bộ phù hợp để đáp ứng hiệu năng và khả năng phục vụ đồng thời nhiều request.
	
	Các công nghệ backend thường gặp có thể thay thế NodeJS gồm Spring Boot (mạnh về cấu trúc và hệ sinh thái enterprise), Django/FastAPI (phát triển nhanh, thuận lợi cho bài toán dữ liệu), hoặc .NET (ổn định và hiệu năng cao). NodeJS được lựa chọn do phù hợp với mục tiêu triển khai một hệ thống web ứng dụng theo hướng dịch vụ trong phạm vi đồ án, đồng thời đồng nhất ngôn ngữ với frontend (JavaScript/TypeScript), giúp tăng tốc độ phát triển, giảm thời gian học công nghệ, và dễ tích hợp các thư viện phục vụ nghiệp vụ như gửi email, xác thực JWT, và kết nối CSDL thông qua ORM.
	
	\section{MySQL}
	MySQL được sử dụng để lưu trữ dữ liệu nghiệp vụ có cấu trúc và có quan hệ rõ ràng, bao gồm: tài khoản, phân quyền, hồ sơ ứng viên/nhà tuyển dụng, tin tuyển dụng, kỹ năng/tag, và các quan hệ nhiều-nhiều giữa hồ sơ và danh mục (ví dụ: ứng viên--kỹ năng, tin tuyển dụng--kỹ năng, ứng viên--tag, tin tuyển dụng--tag). Đây là dạng dữ liệu phù hợp với mô hình cơ sở dữ liệu quan hệ và cần đảm bảo tính nhất quán khi cập nhật (đặc biệt ở các thao tác tạo/sửa hồ sơ, tạo/sửa tin, và ghi nhận lịch sử đề xuất hoặc trạng thái tương tác).
	
	Về lựa chọn thay thế, hệ thống có thể dùng PostgreSQL (mạnh về truy vấn nâng cao và mở rộng dữ liệu kiểu JSON), MongoDB (linh hoạt dạng document, phù hợp dữ liệu ít ràng buộc), hoặc SQLite (đơn giản cho demo nhỏ). MySQL được lựa chọn vì phổ biến, dễ triển khai, đáp ứng tốt yêu cầu CRUD và quan hệ dữ liệu của bài toán tuyển dụng trong phạm vi đồ án, đồng thời phù hợp với mục tiêu xây dựng một hệ thống ổn định, dễ vận hành và dễ mở rộng theo hướng tối ưu chỉ mục hoặc tách lớp truy cập dữ liệu khi cần.
	
	\section{Phương pháp đề xuất dựa theo nội dung (Content-based Filtering -- CBF)}
	Hệ gợi ý (Recommender Systems) là một dạng hệ thống lọc thông tin nhằm dự đoán mức độ quan tâm hoặc mức độ phù hợp giữa người dùng và một đối tượng (item) mà người dùng chưa trực tiếp xem xét trong quá khứ. Trong bối cảnh đồ án tuyển dụng, ``người dùng'' có thể là ứng viên hoặc nhà tuyển dụng, còn ``item'' có thể là tin tuyển dụng hoặc ứng viên. Do đó, bài toán đề xuất của hệ thống được hiểu là bài toán ước lượng mức độ phù hợp giữa các cặp đối tượng để từ đó xếp hạng và chọn ra Top-N kết quả gợi ý cho từng ngữ cảnh sử dụng [9].
	
	Đồ án lựa chọn phương pháp gợi ý dựa theo nội dung (Content-based Filtering -- CBF). Ý tưởng cốt lõi của CBF là sử dụng chính các thuộc tính mô tả của đối tượng để xây dựng ``hồ sơ nội dung'' và thực hiện so khớp dựa trên mức độ tương đồng giữa các hồ sơ đó [9]. Thay vì phụ thuộc mạnh vào lịch sử tương tác dày đặc, CBF tập trung vào dữ liệu mô tả như kỹ năng, lĩnh vực/tag, mức lương và địa điểm, từ đó vẫn có thể đưa ra gợi ý ngay cả khi hệ thống mới vận hành hoặc lượng phản hồi hành vi còn hạn chế.
	
	Trong hệ thống của đồ án, CBF được triển khai theo hướng biểu diễn các thực thể dưới dạng vector đặc trưng (vector hồ sơ). Cụ thể, tin tuyển dụng/nhà tuyển dụng được mô hình hóa bởi các thuộc tính phản ánh nhu cầu tuyển dụng như bộ kỹ năng (kèm mức độ quan trọng), tập tag ngành nghề/lĩnh vực (kèm trọng số), mức lương đại diện và vị trí; ứng viên được mô hình hóa bởi các thuộc tính phản ánh năng lực và mong muốn như kỹ năng đã có, lĩnh vực quan tâm, mức lương kỳ vọng và vị trí ưu tiên. Từ hai vector hồ sơ, hệ thống tính các điểm thành phần tương ứng (kỹ năng, tag, lương, địa điểm) và tổ hợp chúng thành FitScore theo trọng số cho từng ngữ cảnh. Cách làm này phù hợp với yêu cầu đề xuất hai chiều trong tuyển dụng, đồng thời hỗ trợ khả năng giải thích kết quả gợi ý thông qua các thông tin trùng khớp/thiếu hụt (ví dụ: kỹ năng bắt buộc còn thiếu, tag trùng nhau, mức lương cao hơn/thấp hơn kỳ vọng, địa điểm phù hợp hay trung tính).
	
	Xét về các hướng tiếp cận thay thế, hệ thống có thể sử dụng Collaborative Filtering (CF) dựa trên sự tương đồng hành vi giữa người dùng và/hoặc đối tượng, thường khai thác ma trận user--item (ví dụ: click, lưu tin, ứng tuyển, đánh giá) [9]. Ngoài ra, các mô hình lai (Hybrid) cũng là những hướng mạnh khi có đủ dữ liệu phản hồi để học tối ưu thứ hạng gợi ý theo thời gian. Tuy nhiên, trong phạm vi đồ án, CBF được lựa chọn vì ba lý do chính: thứ nhất, giảm ảnh hưởng của bài toán cold-start do không bắt buộc phải có lịch sử tương tác dày; thứ hai, phù hợp với dữ liệu nghiệp vụ sẵn có của tuyển dụng (kỹ năng, tag, lương, địa điểm) và dễ tích hợp vào luồng nghiệp vụ hiện tại; thứ ba, kết quả có thể giải thích rõ ràng, là yếu tố quan trọng trong tuyển dụng vì người dùng thường cần hiểu lý do phù hợp trước khi quyết định ứng tuyển hoặc liên hệ.
	
	Tóm lại, phương pháp CBF trong đồ án được hiện thực hóa bằng cơ chế vector hóa nội dung và công thức FitScore theo ngữ cảnh, qua đó đáp ứng yêu cầu đề xuất cốt lõi của hệ thống, đảm bảo tính khả thi triển khai và tính minh bạch khi trình bày kết quả gợi ý.
	
\end{document}